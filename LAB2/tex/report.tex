\documentclass[12pt, a4paper]{article}

\usepackage[utf8]{inputenc}
\usepackage[T2A]{fontenc}
\usepackage[russian]{babel}
\usepackage[oglav,spisok,boldsect,eqwhole,figwhole,hyperref,hyperprint]{./style/fn2kursstyle}
%\usepackage{listings} 
%\usepackage{courier}
\usepackage{enumitem}
\usepackage{multirow}
\usepackage{hhline}
\usepackage{wrapfig}
\usepackage{amsfonts}

\usepackage{color} %% это для отображения цвета в коде
\usepackage{listings} %% собственно, это и есть пакет listings

\usepackage{caption}
\DeclareCaptionFont{white}{\color{white}} %% это сделает текст заголовка белым
%% код ниже нарисует серую рамочку вокруг заголовка кода.
\DeclareCaptionFormat{listing}{\colorbox{gray}{\parbox{\textwidth}{#1#2#3}}}
\captionsetup[lstlisting]{format=listing,labelfont=white,textfont=white}

\setcounter{page}{1} % начать нумерацию с номера три

\newcommand\Rg{\mathop{\mathrm{Rg}}}

\begin{document}
	%\tableofcontents
	
	\newpage
	
	\section-{Контрольные вопросы}	
	\begin{enumerate}
		\item  \textbf{Вопрос.}  Почему условие $ \|C\| < 1 $ гарантирует сходимость итерационных методов?
		
	 \textbf{Ответ.} Пусть $ \|C\| < 1 $. Покажем, что итерационный метод сходится.
	 
	  \textbf{Определение.} Метод называется сходящимся, если $ \| x - x^{n} \| \rightarrow 0 \text{ при } n \rightarrow \infty$.
		\begin{equation}
			x = C x + y,
			\label{Eq1}
		\end{equation}
		где $ C $ --- квадратная матрица размера $ n \times n $; $ y $ --- вектор столбец.
		
		Запишем рекуррентное соотношение:
		\begin{equation}
			x^{k+1} = C x^k + y,\; \forall k = 0,\, 1,\, 2,\, \dots .
			\label{Eq2}
		\end{equation}
		
		Вычитаем из соотношения \eqref{Eq1} соотношение \eqref{Eq2}, получаем
		
		\begin{equation}
			x - x^{k + 1} = C (x - x^k).
			\label{Eq3}
		\end{equation}
		
		Вычисляя норму левой и правой части этого равенства имеем:
		$$ \| x - x^{k + 1} \| = \| C (x - x^k) \| \leq \| C \| \, \| (x - x^k) \|, \; \forall k = 0,\, 1,\, 2,\, \dots , $$
		
		так как это неравенство верно для всех $k$, то
		$$ \| x - x^{n} \| \leq \| C \| \, \| x - x^{n - 1}\| \leq \| C \|^2 \, \| x - x^{n - 2} \| \leq \dots \leq \| C \|^n \, \| x - x^0 \|. $$
		
		Норма $ \| x - x^0 \|$ не зависит от $n$. Используя условие $\| C \| < 1$, получаем:
		$$ \|C\|^n \rightarrow 0 \text{ при } n \rightarrow \infty \Longrightarrow \| x - x^{n} \| \rightarrow 0 \text{ при } n \rightarrow \infty. $$ 
		
	 Таким образом, метод сходится.
	 
	 	\item  \textbf{Вопрос.}  Каким следует выбирать итерационный параметр $ \tau $ в методе простой итерации для увеличения скорости сходимости? Как выбрать начальное приближение $ x^0 $?
	 	
	 	Обычно для улучшения скорости сходимости исходную систему,
	 	прежде чем приводить к виду, удобному для итераций, умножают
	 	на итерационный параметр $\tau$, который выбирают так, чтобы
	 	выполнялась оценка $\|C\| \leq 1$ и норма матрицы C была как можно
	 	меньше. Однако мы не можем выбирать параметр слишком малым, поскольку тогда погрешность вычислений станет слишком большой. Начальной значение $x^0$ стоит выбирать как можно более близкое к решению, если это возможно.
	 	
		\item  \textbf{Вопрос.} На примере системы из двух уравнений с двумя неизвестными дайте геометрическую интерпретацию метода простой итерации, метода Якоби, метода Зейделя, метода релаксации.
		
		Геометрическая интерпретация метода простой итерации для скалярного случая   приведена на рис. 1. Алгоритм метода простых итераций таков.
		
		1. Локализуем корень, приближенно определяем, на каком отрезке он находится. Вопрос локализации корня не решается алгоритмически, это скорее вопрос искусства вычислителя, хотя во многих случаях локализовать корень достаточно легко.
		2. Выбираем точку u0 на оси 0u
		3. Вычисляем F(u0)
		4. Определяем точку u1 по значению F(u0):
		4.1 Пересечение горизонтальной прямой AA’ с прямой v = u есть точка С (ОА = v1, AC = u1)
		4.2. Очевидно, что горизонтальная координата точки С и есть u1 (так как F(u0) = u1).
		4.3. Опустим перпендикуляр из С на 0u. Поскольку ОА = u1, то u1 — значение на первой итерации.
		5. Аналогично строим точки u2, u3
		
	\item  \textbf{Вопрос.} При каких условиях сходятся метод простой итерации, метод Якоби, метод Зейделя и метод релаксации? Какую матрицу называют положительно определенной?
	
	В данном случае мы будем рассматривать норму изменения за итерацию, а не норму поргешности численного решения. Если сходимость будем происходить очень медленно, то использование такой оценки для точки останова приведет к неверному ответу.
	
	\textbf{Ответ.}
	 \textbf{Определение.} Линейный оператор $ A $, действующий в линейном пространстве $ H $, называется:
	\begin{enumerate}
		\item положительным, если $ \forall \, x \in H, \; x \ne 0 \; (Ax,x) > 0; $
		\item положительно определенным, если $ \exists \, \delta > 0: \; \forall \, x \in H \; (Ax,x) \geq \delta (x,x)$.
	\end{enumerate}
\medskip

Если матрица $ А $ симметричная положительно определенная и $ B - 0.5\tau A > 0$ и $\tau > 0 $, то стационарный итерационный метод $ B \frac{x^{k + 1} - x^k}{\tau}+A x^k = f $ сходится.

Метод простой итерации сходится, если $ \tau < \frac{2}{\lambda_{\text{max}}} $, где $\lambda_{\text{max}}$ --- максимальное собственное значение симметричной положительно определенной матрицы $A$.

 Метод Якоби сходится, если матрица $A$ --- симметричная положительно определенная матрица с диагональным преобладанием.
 
 Метод релаксации сходится, если $ 0 < \omega < 2 $, $\omega$ --- заданный числовой параметр (параметр релаксации).
 
  Метод Зейделя частный случай метода релаксации при $ \omega = 1 $.
  
  	\item  \textbf{Вопрос.}	Выпишите матрицу C для методов Зейделя и релаксации.
  	
  \textbf{Ответ.}	Каноническая форма метода релаксации:
  	$$	\left( D + \omega L \right) \frac{x^{k + 1} - x^k}{\omega} + A x^k = b;\quad \forall k = 0,\, 1,\, 2,\, \dots,$$ где $\omega$ --- заданный числовой параметр (параметр релаксации).
  	
  	 $ A = L + D + U,$ где $L$ --- нижняя треугольная матрица, $D$ --- диагональная матрица, $U$ --- верхняя треугольная матрица.
  	
  	Домножим на $\omega$ и перенесем все слагаемые кроме $x^{k + 1}$ вправо:
  	$$\left( D + \omega L \right)x^{k + 1} = \left( D + \omega L -\omega A \right)x^k +\omega b; \quad \forall k = 0,\, 1,\, 2,\, \dots. $$
  	
  	Домножим обе части на $ \left( D + \omega L \right)^{-1}$ слева:
  		$$x^{k + 1} = \left( D + \omega L \right)^{-1} \left( D + \omega L -\omega A \right)x^k +\omega \left( D + \omega L \right)^{-1} b. $$
  	Получаем:
  	$$ C = \left( D + \omega L \right)^{-1} \left( D + \omega L -\omega A \right) = \left( D + \omega L \right)^{-1} \left( \left( 1 - \omega \right) D  -\omega U \right).  $$
  	
  	Для метода Зейделя $\omega = 1$ и матрица $C$ принимает вид:
  	$$ C = - \left( D + L \right)^{-1}U.  $$ 
  	
 	\item  \textbf{Вопрос.} Почему в общем случае для остановки итерационного процесса нельзя использовать критерий $\| x^k - x^{k-1} \| < \varepsilon$?
 	
 	\item  \textbf{Вопрос.} Какие еще критерии окончания итерационного процесса Вы можете предложить?
 	
 	 \textbf{Ответ.}
 	\begin{gather*}
 		\| x^k - x^{k - 1} \| \leq \| x^k\| \varepsilon + \varepsilon_0; \\
 		\left\| \frac{ x^k - x^{k - 1}}{\| x^k \| + \varepsilon_0} \right\| \leq \varepsilon; \\
 		 		\| A x^{k + 1} - f \| \leq \varepsilon; \\
 		\| x^k - x^{k - 1}\| \leq \frac{1 - \| C \|}{\| C \|} \varepsilon. 
 	\end{gather*}
 	Для метода Зейделя:
 	$$\| x^k - x^{k - 1} \| \leq \frac{1-\| C \|}{\| C_U \|} \varepsilon.$$
 	
  	
  	
		\end{enumerate}

	\end{document}