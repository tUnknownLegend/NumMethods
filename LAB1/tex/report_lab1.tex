\documentclass{article}

\usepackage[utf8]{inputenc}
\usepackage[T2A]{fontenc}
\usepackage[russian]{babel}
\usepackage{amsfonts}
\usepackage{amssymb}
\usepackage{amsmath}
\usepackage{graphicx}
\usepackage{courier}


\begin{document}
 \textbf{Вопрос№2.}  Докажите, что если $\det A \ne 0$, то при выборе главного элемента в столбце среди элементов, лежащих не выше главной диагонали, всегда найдется хотя бы один элемент, отличный от нуля. 
 
 \textbf{Ответ.} Доказательство от противного. 
 Предположим, что  на $i-$ом шаге при выборе главного элемента в $i-$ом столбце все элементы не выше главной диагонали нулевые, то есть матрица имеет вид:
 \[ A^{(i)} = 
 \begin{pmatrix}
 	a_{11} & a_{12} & \ldots  & a_{1i} & a_{1(i+1)} & \ldots & a_{1n}\\
 	0 & a_{22} & \ldots & a_{2i} & a_{2(i+1)} & \ldots & a_{2n}\\
 	\hdotsfor[2.5]{7}\\
 	0 & 0 & \ldots & a_{(i-1)i} & a_{(i-1)(i+1)} & \ldots & a_{(i-1)n}\\
 	0 & 0 & \ldots &  0 & a_{i(i+1)} & \ldots & a_{in}\\
 	0 & 0 & \ldots &  0 & a_{(i+1)(i+1)} & \ldots & a_{in}\\
 	\hdotsfor[2.5]{7}\\
 	0 & 0 & \ldots & 0  & a_{n(i+1)} & \ldots & a_{nn}\\
 \end{pmatrix}
 \].
 
 Элементарные преобразования строк (столбцов) матрицы не меняют значения определителя, следовательно, определитель полученной матрицы $A^{(i)}$ равен определителю исходной марицы $A$.
 
  Теперь раскроем определитель $A^{(i)}$ по первому столбцу $(i-1)$ раз. В итоге получим:
 $$\det A^{(i)} = a_{11}  \cdot \ldots \cdot a_{(i-1)(i-1)} \cdot 
 \begin{vmatrix}
 	0 & a_{i(i+1)} & \ldots & a_{in}\\
 	0 & a_{(i+1)(i+1)} & \ldots & a_{in}\\
 	\hdotsfor[2.5]{4}\\
 	0 & a_{n(i+1)} & \ldots & a_{nn}\\
 \end{vmatrix} = 0,
 $$
 но по условию $\det A \ne 0$. Получили противоречие.
 Доказано
 
 \medskip
 
 \textbf{Вопрос№5.} А) Приведите пример матрицы, у которой число обусловленности велико, а определитель мал. 
 
 Рассмотрим такую матрицу:
 \[ 
 \begin{pmatrix}
 	1 & 0\\
 	0 & 10^{-5}
 \end{pmatrix}.
\]
Определитель $\det A = 10^{-5} $. Число обусловленности $cond A = 10^5$.

 Б) Приведите пример матрицы, у которой число обусловленности мало, а определитель велик.
 
 Пусть значение $\varepsilon$ близко к нулю.
 
 Рассмотрим такую матрицу:
 \[
 \begin{pmatrix}
 	\frac{1}{\varepsilon} & 0\\
 	0 & \frac{1}{\varepsilon}
 \end{pmatrix}.
 \]
 
 Ее определитель равен бесконечности.
 
 Теперь домножим матрицу на $\varepsilon$.
 
У числа обусловленности есть свойство: умножение матрицы $A$ на произвольную
константу $\alpha \ne 0$ не приведет к изменению ее числа обусловленности, т. к. в этом случае обратная матрица окажется умноженной
на величину $\alpha^{-1}$.

Число обусловленности $cond A = 1$.

\textbf{Вопрос№8.} В каких случаях целесообразно использовать метод Гаусса, а в каких — методы, основанные на факторизации матрицы?

Метод Гаусса целесообразно использовать, когда матрица и столбец изменяются. 

Метод $QR$-разложения целесообразно использовать, когда матрица остается неизменной, а столбец правой части изменяется. С помощью метода вращений можно один раз вычислить ортогональную матрицу $Q$ и верхнетреугольную матрицу $R$, а далее использовать только обратный ход метода Гаусса для различных векторов правой части.

 \textbf{Вопрос№10.} Объясните, почему, говоря о векторах, норму $\|\cdot\|_1$ часто называют октаэдрической, норму $\|\cdot\|_2$ — шаровой, а норму $\|\cdot\|_{\infty}$ — кубической. 

Норму $\|\cdot\|_1$ называют октаэдрической, т.к. единичный шар ${\{ x: \|x\|_1 \leq 1\}}$ представляет собой в трехмерном пространстве октаэдр. Норму $\|\cdot\|_2$ называют шаровой, так как единичный шар $\{ x: \|x\|_2 \leq 1\}$ представляет собой в трехмерном пространстве шар. Норму $\|\cdot\|_{\infty}$ называют кубической, так как единичный шар $\{ x: \|x\|_{\infty} \leq 1\}$ представляет собой в трехмерном пространстве куб.


\end{document}