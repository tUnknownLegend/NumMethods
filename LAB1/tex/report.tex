\documentclass[12pt, a4paper]{article}

\usepackage[utf8]{inputenc}
\usepackage[T2A]{fontenc}
\usepackage[russian]{babel}
\usepackage[oglav,spisok,boldsect,eqwhole,figwhole,hyperref,hyperprint]{./style/fn2kursstyle}
%\usepackage{listings} 
%\usepackage{courier}
\usepackage{enumitem}
\usepackage{multirow}
\usepackage{hhline}
\usepackage{wrapfig}
\usepackage{amsfonts}

\usepackage{color} %% это для отображения цвета в коде
\usepackage{listings} %% собственно, это и есть пакет listings

\usepackage{caption}
\DeclareCaptionFont{white}{\color{white}} %% это сделает текст заголовка белым
%% код ниже нарисует серую рамочку вокруг заголовка кода.
\DeclareCaptionFormat{listing}{\colorbox{gray}{\parbox{\textwidth}{#1#2#3}}}
\captionsetup[lstlisting]{format=listing,labelfont=white,textfont=white}

\setcounter{page}{1} % начать нумерацию с номера три

\newcommand\Rg{\mathop{\mathrm{Rg}}}

\begin{document}
	%\tableofcontents
	
	\newpage
	
\section-{Контрольные вопросы}	
\begin{enumerate}
	\item 
	\item
	\item \textbf{Вопрос.} В методе Гаусса с полным выбором ведущего элемента
	приходится не только переставлять уравнения, но и менять нумерацию неизвестных. Предложите алгоритм, позволяющий восстановить первоначальный порядок неизвестных.
	
	\textbf{Ответ.} Можно создать два массива, один из которых будет отвечать за порядок строк, а второй - за порядок столбцов. Если размер изначальной матрицы $N \times N$, длинна массивов будет $N$. Изначально массивы заполнены числами от 0 до $N - 1$, расположенных по возрастанию. Во время исполнения метода полного перебора, одновременно с переменой мест строк или столбцов будем менять элементы соответствующих массивов. Таким образом мы сохраним исходные номера строк и столбцов. Тогда, если мы будем обращаться к элементам введенных массивов по изначальным индексам, то значение элемента массива будет соответствовать новому индексу элемента матрицы.
	\item
	\item
	
	\item 
	Как упрощается оценка числа обусловленности, если матрица является:
		\begin{enumerate}
			\item \textbf{диагональной}. $cond A = \|A^{-1}\|$ $\| A \|$. Поскольку обратная для  диагональной матрица обратная - это диагональная со всеми элементами в -1 степени, то расчеты сильно сокращаются. В связи с тем, что все элементы матрицы - собственные числа, то можно легко получить оценку снизу, ей будет отношение максимального элемента к минимальному, взятых по модулю
			\item \textbf{симметричной}. Рассмотрим симметричную матрицу $A$. $A^-1$ тоже симметричная. Следовательно, можно найти только половину диагональных элементов выше или ниже главной диагонали, так как остальные будут такие же.
			\item \textbf{ортогональной}. Поскольку $A^{-1} = A$, то $cond A = 1$
			\item \textbf{положительно определенной}. Если матрица $A$ положительная определена, то она не вырождена, так как по критерию Сильвестра $det(A) > 0$. Кроме того все собственные числа будут положительные.
			\item \textbf{треугольной}. У треугольной матрицы, элементы расположенные на диагонали - собственные числа, поэтому оценку снизу можно получить аналогично с пунктом а.
		\end{enumerate}
	\item \textbf{Вопрос.} Применимо ли понятие числа обусловленности к вырож-
	денным матрицам?
	
	\textbf{Ответ.} Если умножить вырожденную матрицу на вектор, то получаем нулевой вектор. $M = \|A\| = max \frac{\|Ax\|}{\|x\|}$, $m = min \frac{\|Ax\|}{\|x\|}$, $k(A) = \frac{M}{m}$. Для вырожденной матрицы $m = 0$, обратной матрицы не существует, поэтому $cond = infinty$~\cite{conditionNum}.
	\item  \textbf{Вопрос.} В каких случаях целесообразно использовать метод Гаусса,
	а в каких — методы, основанные на факторизации матрицы?
	
	\textbf{Ответ.} Метод гаусса удобно использовать, когда исходная матрица треугольная или близка к треугольной. QR метод удобен, когда изначальная матрица ортогональная. Рассмотрим СЛАУ $Ax = b$. Если изменяется только вектор $b$, то QR метод будет иметь преимущество над методом Гаусса, поскольку матрица результирующего вращения $T$ не будет изменяться, а значит останутся постоянными и матрицы $Q$, $R$.
	В общем случае QR метод требует значительно большего числа операций, чем метод Гаусса, поэтому метода Гаусса будет быстрее~\cite{MetVich}.
	\item \textbf{Вопрос.} Как можно объединить в одну процедуру прямой и обратный ход метода Гаусса? В чем достоинства и недостатки
	такого подхода?
	
	\textbf{Ответ.} Рассмотрим СЛАУ $Ax = b$. Обнуляем коэффициенты $a_{ii}$ под главной диагональю, затем обнуляем элементы над главной диагональю, ответом будет вектор с элементами вида $\frac {b_i}{a_{ii}}$. Преимущество заключается в том, что мы сделаем меньше итераций, благодаря объедению работы прямого и обратного метода в один цикл. Недостаток заключается в нарушении принципа единственной ответственности, что несет в себе: 
	\begin{itemize}
		\item отсутствие возможности использовать методы прямого и обратного обхода раздельно.
		\item ухудшение тестируемости кода, а значит потенциальные проблемы при отладке и внесении изменений в программу.
	\end{itemize}
	\item \textbf{Вопрос.} Объясните, почему, говоря о векторах, норму $\|\cdot\|_1$ часто
	называют октаэдрической, норму $\|\cdot\|_2$ — шаровой, а норму $\|\cdot\|_\infty$
	— кубической.
	
	\textbf{Ответ:}
		\begin{enumerate}
		\item Октаэдрическая норма $\|\cdot\|_1$ вектора $x$ в $\mathbb{R}^3$ на единичном шаре будет октаэдром.
		\item Шаровая норма $\|\cdot\|_2$ вектора $x$ в $\mathbb{R}^3$ на единичном шаре будет шаром.
		\item Кубическая норма $\|\cdot\|_\infty$ вектора $x$ в $\mathbb{R}^3$ на единичном шаре будет кубом.
		\end{enumerate}
\end{enumerate}

%__________________________________________________________________________________
% Библиотека

\newpage

\begin{thebibliography}{5}
	\bibitem{MetVich} Численные методы решения задач линейной алгебры:
	методические указания к выполнению лабораторных работ по
	курсу «Методы вычислений» / И. К. Марчевский, О. В. Щерица; под ред. М. П. Галанина. — Москва : Издательство МГТУ
	им. Н. Э. Баумана, 2017. — 59, [1] с.
	\bibitem{conditionNum}
	What is the condition number of a matrix? // phys.uconn.edu.\\
	URL: https://www.phys.uconn.edu/~rozman/Courses/m3511\_18s/downloads/condnumber.pdf \\
	(дата обращения: 18.09.2022).
	
\end{thebibliography}

\end{document}